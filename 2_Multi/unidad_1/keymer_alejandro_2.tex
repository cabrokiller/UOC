\documentclass[]{article}
\usepackage{lmodern}
\usepackage{amssymb,amsmath}
\usepackage{ifxetex,ifluatex}
\usepackage{fixltx2e} % provides \textsubscript
\ifnum 0\ifxetex 1\fi\ifluatex 1\fi=0 % if pdftex
  \usepackage[T1]{fontenc}
  \usepackage[utf8]{inputenc}
\else % if luatex or xelatex
  \ifxetex
    \usepackage{mathspec}
  \else
    \usepackage{fontspec}
  \fi
  \defaultfontfeatures{Ligatures=TeX,Scale=MatchLowercase}
\fi
% use upquote if available, for straight quotes in verbatim environments
\IfFileExists{upquote.sty}{\usepackage{upquote}}{}
% use microtype if available
\IfFileExists{microtype.sty}{%
\usepackage{microtype}
\UseMicrotypeSet[protrusion]{basicmath} % disable protrusion for tt fonts
}{}
\usepackage[margin=1in]{geometry}
\usepackage{hyperref}
\hypersetup{unicode=true,
            pdftitle={Análisis Multivariante - Ejercicios 1.2},
            pdfauthor={Alejandro Keymer},
            pdfborder={0 0 0},
            breaklinks=true}
\urlstyle{same}  % don't use monospace font for urls
\usepackage{color}
\usepackage{fancyvrb}
\newcommand{\VerbBar}{|}
\newcommand{\VERB}{\Verb[commandchars=\\\{\}]}
\DefineVerbatimEnvironment{Highlighting}{Verbatim}{commandchars=\\\{\}}
% Add ',fontsize=\small' for more characters per line
\usepackage{framed}
\definecolor{shadecolor}{RGB}{248,248,248}
\newenvironment{Shaded}{\begin{snugshade}}{\end{snugshade}}
\newcommand{\AlertTok}[1]{\textcolor[rgb]{0.94,0.16,0.16}{#1}}
\newcommand{\AnnotationTok}[1]{\textcolor[rgb]{0.56,0.35,0.01}{\textbf{\textit{#1}}}}
\newcommand{\AttributeTok}[1]{\textcolor[rgb]{0.77,0.63,0.00}{#1}}
\newcommand{\BaseNTok}[1]{\textcolor[rgb]{0.00,0.00,0.81}{#1}}
\newcommand{\BuiltInTok}[1]{#1}
\newcommand{\CharTok}[1]{\textcolor[rgb]{0.31,0.60,0.02}{#1}}
\newcommand{\CommentTok}[1]{\textcolor[rgb]{0.56,0.35,0.01}{\textit{#1}}}
\newcommand{\CommentVarTok}[1]{\textcolor[rgb]{0.56,0.35,0.01}{\textbf{\textit{#1}}}}
\newcommand{\ConstantTok}[1]{\textcolor[rgb]{0.00,0.00,0.00}{#1}}
\newcommand{\ControlFlowTok}[1]{\textcolor[rgb]{0.13,0.29,0.53}{\textbf{#1}}}
\newcommand{\DataTypeTok}[1]{\textcolor[rgb]{0.13,0.29,0.53}{#1}}
\newcommand{\DecValTok}[1]{\textcolor[rgb]{0.00,0.00,0.81}{#1}}
\newcommand{\DocumentationTok}[1]{\textcolor[rgb]{0.56,0.35,0.01}{\textbf{\textit{#1}}}}
\newcommand{\ErrorTok}[1]{\textcolor[rgb]{0.64,0.00,0.00}{\textbf{#1}}}
\newcommand{\ExtensionTok}[1]{#1}
\newcommand{\FloatTok}[1]{\textcolor[rgb]{0.00,0.00,0.81}{#1}}
\newcommand{\FunctionTok}[1]{\textcolor[rgb]{0.00,0.00,0.00}{#1}}
\newcommand{\ImportTok}[1]{#1}
\newcommand{\InformationTok}[1]{\textcolor[rgb]{0.56,0.35,0.01}{\textbf{\textit{#1}}}}
\newcommand{\KeywordTok}[1]{\textcolor[rgb]{0.13,0.29,0.53}{\textbf{#1}}}
\newcommand{\NormalTok}[1]{#1}
\newcommand{\OperatorTok}[1]{\textcolor[rgb]{0.81,0.36,0.00}{\textbf{#1}}}
\newcommand{\OtherTok}[1]{\textcolor[rgb]{0.56,0.35,0.01}{#1}}
\newcommand{\PreprocessorTok}[1]{\textcolor[rgb]{0.56,0.35,0.01}{\textit{#1}}}
\newcommand{\RegionMarkerTok}[1]{#1}
\newcommand{\SpecialCharTok}[1]{\textcolor[rgb]{0.00,0.00,0.00}{#1}}
\newcommand{\SpecialStringTok}[1]{\textcolor[rgb]{0.31,0.60,0.02}{#1}}
\newcommand{\StringTok}[1]{\textcolor[rgb]{0.31,0.60,0.02}{#1}}
\newcommand{\VariableTok}[1]{\textcolor[rgb]{0.00,0.00,0.00}{#1}}
\newcommand{\VerbatimStringTok}[1]{\textcolor[rgb]{0.31,0.60,0.02}{#1}}
\newcommand{\WarningTok}[1]{\textcolor[rgb]{0.56,0.35,0.01}{\textbf{\textit{#1}}}}
\usepackage{graphicx,grffile}
\makeatletter
\def\maxwidth{\ifdim\Gin@nat@width>\linewidth\linewidth\else\Gin@nat@width\fi}
\def\maxheight{\ifdim\Gin@nat@height>\textheight\textheight\else\Gin@nat@height\fi}
\makeatother
% Scale images if necessary, so that they will not overflow the page
% margins by default, and it is still possible to overwrite the defaults
% using explicit options in \includegraphics[width, height, ...]{}
\setkeys{Gin}{width=\maxwidth,height=\maxheight,keepaspectratio}
\IfFileExists{parskip.sty}{%
\usepackage{parskip}
}{% else
\setlength{\parindent}{0pt}
\setlength{\parskip}{6pt plus 2pt minus 1pt}
}
\setlength{\emergencystretch}{3em}  % prevent overfull lines
\providecommand{\tightlist}{%
  \setlength{\itemsep}{0pt}\setlength{\parskip}{0pt}}
\setcounter{secnumdepth}{0}
% Redefines (sub)paragraphs to behave more like sections
\ifx\paragraph\undefined\else
\let\oldparagraph\paragraph
\renewcommand{\paragraph}[1]{\oldparagraph{#1}\mbox{}}
\fi
\ifx\subparagraph\undefined\else
\let\oldsubparagraph\subparagraph
\renewcommand{\subparagraph}[1]{\oldsubparagraph{#1}\mbox{}}
\fi

%%% Use protect on footnotes to avoid problems with footnotes in titles
\let\rmarkdownfootnote\footnote%
\def\footnote{\protect\rmarkdownfootnote}

%%% Change title format to be more compact
\usepackage{titling}

% Create subtitle command for use in maketitle
\providecommand{\subtitle}[1]{
  \posttitle{
    \begin{center}\large#1\end{center}
    }
}

\setlength{\droptitle}{-2em}

  \title{Análisis Multivariante - Ejercicios 1.2}
    \pretitle{\vspace{\droptitle}\centering\huge}
  \posttitle{\par}
    \author{Alejandro Keymer}
    \preauthor{\centering\large\emph}
  \postauthor{\par}
    \date{}
    \predate{}\postdate{}
  

\begin{document}
\maketitle

\hypertarget{matrices}{%
\section{1. Matrices}\label{matrices}}

\hypertarget{las-siguientes-cuestiones-se-refieren-a-propiedades-elementales-del-algebra-matricial-que-debesconocer.-indicar-si-son-ciertas-o-falsas-las-siguientes-propiedades}{%
\paragraph{1. Las siguientes cuestiones se refieren a propiedades
elementales del álgebra matricial que debesconocer. Indicar si son
CIERTAS o FALSAS las siguientes
propiedades}\label{las-siguientes-cuestiones-se-refieren-a-propiedades-elementales-del-algebra-matricial-que-debesconocer.-indicar-si-son-ciertas-o-falsas-las-siguientes-propiedades}}

\begin{Shaded}
\begin{Highlighting}[]
\NormalTok{A <-}\StringTok{ }\KeywordTok{matrix}\NormalTok{(}\KeywordTok{c}\NormalTok{(}\DecValTok{1}\NormalTok{,}\DecValTok{2}\NormalTok{,}\DecValTok{1}\NormalTok{,}\DecValTok{5}\NormalTok{,}\DecValTok{4}\NormalTok{,}\DecValTok{4}\NormalTok{,}\DecValTok{2}\NormalTok{,}\DecValTok{0}\NormalTok{,}\DecValTok{2}\NormalTok{), }\DataTypeTok{nrow =} \DecValTok{3}\NormalTok{)}
\NormalTok{B <-}\StringTok{ }\KeywordTok{matrix}\NormalTok{(}\KeywordTok{c}\NormalTok{(}\DecValTok{1}\NormalTok{,}\DecValTok{3}\NormalTok{,}\DecValTok{0}\NormalTok{,}\DecValTok{1}\NormalTok{,}\DecValTok{0}\NormalTok{,}\DecValTok{1}\NormalTok{,}\DecValTok{1}\NormalTok{,}\DecValTok{1}\NormalTok{,}\DecValTok{5}\NormalTok{), }\DataTypeTok{nrow =} \DecValTok{3}\NormalTok{)}
\NormalTok{D <-}\StringTok{ }\KeywordTok{matrix}\NormalTok{(}\KeywordTok{c}\NormalTok{(}\DecValTok{1}\NormalTok{,}\DecValTok{2}\NormalTok{,}\DecValTok{3}\NormalTok{,}\DecValTok{5}\NormalTok{,}\DecValTok{4}\NormalTok{,}\DecValTok{3}\NormalTok{), }\DataTypeTok{nrow =} \DecValTok{2}\NormalTok{)}

\CommentTok{# a) VERDADERO}
\KeywordTok{all.equal}\NormalTok{(}\KeywordTok{t}\NormalTok{(A}\OperatorTok{+}\NormalTok{B), }\KeywordTok{t}\NormalTok{(A) }\OperatorTok{+}\StringTok{ }\KeywordTok{t}\NormalTok{(B))}
\end{Highlighting}
\end{Shaded}

\begin{verbatim}
## [1] TRUE
\end{verbatim}

\begin{Shaded}
\begin{Highlighting}[]
\CommentTok{# b) FALSO}
\KeywordTok{all.equal}\NormalTok{(A}\OperatorTok\NormalTok{B, B}\OperatorTok\NormalTok{A)}
\end{Highlighting}
\end{Shaded}

\begin{verbatim}
## [1] "Mean relative difference: 1.079545"
\end{verbatim}

\begin{Shaded}
\begin{Highlighting}[]
\CommentTok{# c) VERDADERO}
\KeywordTok{all.equal}\NormalTok{(}\KeywordTok{det}\NormalTok{(A}\OperatorTok\NormalTok{B), }\KeywordTok{det}\NormalTok{(A)}\OperatorTok{*}\KeywordTok{det}\NormalTok{(B))}
\end{Highlighting}
\end{Shaded}

\begin{verbatim}
## [1] TRUE
\end{verbatim}

\begin{Shaded}
\begin{Highlighting}[]
\CommentTok{# d) VERDADERO}
\KeywordTok{all.equal}\NormalTok{(}\KeywordTok{det}\NormalTok{(A}\OperatorTok\NormalTok{B), }\KeywordTok{det}\NormalTok{(B}\OperatorTok\NormalTok{A))}
\end{Highlighting}
\end{Shaded}

\begin{verbatim}
## [1] TRUE
\end{verbatim}

\begin{Shaded}
\begin{Highlighting}[]
\CommentTok{# e) VERDADERO}
\KeywordTok{all.equal}\NormalTok{(}\KeywordTok{sum}\NormalTok{(}\KeywordTok{diag}\NormalTok{(A }\OperatorTok{+}\StringTok{ }\NormalTok{B)), }\KeywordTok{sum}\NormalTok{(}\KeywordTok{diag}\NormalTok{(A)) }\OperatorTok{+}\StringTok{ }\KeywordTok{sum}\NormalTok{(}\KeywordTok{diag}\NormalTok{(B)))}
\end{Highlighting}
\end{Shaded}

\begin{verbatim}
## [1] TRUE
\end{verbatim}

\begin{Shaded}
\begin{Highlighting}[]
\CommentTok{# f) VERDADERO}
\KeywordTok{all.equal}\NormalTok{(}\KeywordTok{sum}\NormalTok{(}\KeywordTok{diag}\NormalTok{(A}\OperatorTok\NormalTok{B)),}\KeywordTok{sum}\NormalTok{(}\KeywordTok{diag}\NormalTok{(B}\OperatorTok\NormalTok{A)))}
\end{Highlighting}
\end{Shaded}

\begin{verbatim}
## [1] TRUE
\end{verbatim}

\begin{Shaded}
\begin{Highlighting}[]
\CommentTok{# g) VERDADERO}
\KeywordTok{dim}\NormalTok{(D }\OperatorTok\StringTok{ }\KeywordTok{t}\NormalTok{(D))}
\end{Highlighting}
\end{Shaded}

\begin{verbatim}
## [1] 2 2
\end{verbatim}

\begin{Shaded}
\begin{Highlighting}[]
\CommentTok{# h) FALSO}
\KeywordTok{all.equal}\NormalTok{(}\KeywordTok{t}\NormalTok{(A}\OperatorTok\NormalTok{B), }\KeywordTok{t}\NormalTok{(A)}\OperatorTok\KeywordTok{t}\NormalTok{(B))}
\end{Highlighting}
\end{Shaded}

\begin{verbatim}
## [1] "Mean relative difference: 1.079545"
\end{verbatim}

\begin{Shaded}
\begin{Highlighting}[]
\CommentTok{# i) VERDADERO}
\KeywordTok{all.equal}\NormalTok{(}\KeywordTok{solve}\NormalTok{(}\KeywordTok{t}\NormalTok{(A)), }\KeywordTok{t}\NormalTok{(}\KeywordTok{solve}\NormalTok{(A)))}
\end{Highlighting}
\end{Shaded}

\begin{verbatim}
## [1] TRUE
\end{verbatim}

\begin{Shaded}
\begin{Highlighting}[]
\CommentTok{# j)}
\CommentTok{# ?}
\end{Highlighting}
\end{Shaded}

\hypertarget{comprobar-las-siguientes-propiedades}{%
\paragraph{2.(∗)Comprobar las siguientes
propiedades:}\label{comprobar-las-siguientes-propiedades}}

\hypertarget{dadas-las-matrices}{%
\paragraph{3. Dadas las matrices:}\label{dadas-las-matrices}}

\[A = \begin{pmatrix}
2 & 0 & 1 \\
3 & 0 & 0 \\
5 & 1 & 1\\
\end{pmatrix}
B =  \begin{pmatrix}
1 & 0 & 1 \\
1 & 2 & 1 \\
1 & 1 & 0\\
\end{pmatrix}
\]

Calcular:A + B; A − B; AB; BA; A. Hacerlo manualmente y con un programa
como R.

\begin{Shaded}
\begin{Highlighting}[]
\NormalTok{A <-}\StringTok{ }\KeywordTok{matrix}\NormalTok{(}\KeywordTok{c}\NormalTok{(}\DecValTok{2}\NormalTok{,}\DecValTok{3}\NormalTok{,}\DecValTok{5}\NormalTok{,}\DecValTok{0}\NormalTok{,}\DecValTok{0}\NormalTok{,}\DecValTok{1}\NormalTok{,}\DecValTok{1}\NormalTok{,}\DecValTok{0}\NormalTok{,}\DecValTok{1}\NormalTok{), }\DataTypeTok{ncol=}\DecValTok{3}\NormalTok{)}
\NormalTok{B <-}\StringTok{ }\KeywordTok{matrix}\NormalTok{(}\KeywordTok{c}\NormalTok{(}\DecValTok{1}\NormalTok{,}\DecValTok{1}\NormalTok{,}\DecValTok{1}\NormalTok{,}\DecValTok{0}\NormalTok{,}\DecValTok{2}\NormalTok{,}\DecValTok{1}\NormalTok{,}\DecValTok{1}\NormalTok{,}\DecValTok{1}\NormalTok{,}\DecValTok{0}\NormalTok{), }\DataTypeTok{ncol=}\DecValTok{3}\NormalTok{)}

\CommentTok{# a) A + B}
\NormalTok{A }\OperatorTok{+}\StringTok{ }\NormalTok{B}
\end{Highlighting}
\end{Shaded}

\begin{verbatim}
##      [,1] [,2] [,3]
## [1,]    3    0    2
## [2,]    4    2    1
## [3,]    6    2    1
\end{verbatim}

\begin{Shaded}
\begin{Highlighting}[]
\CommentTok{# b) A - B}
\NormalTok{A }\OperatorTok{-}\StringTok{ }\NormalTok{B}
\end{Highlighting}
\end{Shaded}

\begin{verbatim}
##      [,1] [,2] [,3]
## [1,]    1    0    0
## [2,]    2   -2   -1
## [3,]    4    0    1
\end{verbatim}

\begin{Shaded}
\begin{Highlighting}[]
\CommentTok{# c) AB}
\NormalTok{A }\OperatorTok\StringTok{ }\NormalTok{B}
\end{Highlighting}
\end{Shaded}

\begin{verbatim}
##      [,1] [,2] [,3]
## [1,]    3    1    2
## [2,]    3    0    3
## [3,]    7    3    6
\end{verbatim}

\begin{Shaded}
\begin{Highlighting}[]
\CommentTok{# d) BA}
\NormalTok{B }\OperatorTok\StringTok{ }\NormalTok{A}
\end{Highlighting}
\end{Shaded}

\begin{verbatim}
##      [,1] [,2] [,3]
## [1,]    7    1    2
## [2,]   13    1    2
## [3,]    5    0    1
\end{verbatim}

\begin{Shaded}
\begin{Highlighting}[]
\CommentTok{# e) A'}
\KeywordTok{t}\NormalTok{(A)}
\end{Highlighting}
\end{Shaded}

\begin{verbatim}
##      [,1] [,2] [,3]
## [1,]    2    3    5
## [2,]    0    0    1
## [3,]    1    0    1
\end{verbatim}

\hypertarget{demostrar-que-a-a-2i-0-siendo}{%
\paragraph{4. Demostrar que A²− A − 2I = 0,
siendo:}\label{demostrar-que-a-a-2i-0-siendo}}

\[ A =  \begin{pmatrix}
0 & 1 & 1 \\
1 & 0 & 1 \\
1 & 1 & 0\\
\end{pmatrix}
\]

\begin{Shaded}
\begin{Highlighting}[]
\NormalTok{I <-}\StringTok{ }\KeywordTok{diag}\NormalTok{(}\DecValTok{1}\NormalTok{,}\DecValTok{3}\NormalTok{)}
\NormalTok{A <-}\StringTok{ }\KeywordTok{matrix}\NormalTok{(}\KeywordTok{c}\NormalTok{(}\DecValTok{0}\NormalTok{,}\DecValTok{1}\NormalTok{,}\DecValTok{1}\NormalTok{,}\DecValTok{1}\NormalTok{,}\DecValTok{0}\NormalTok{,}\DecValTok{1}\NormalTok{,}\DecValTok{1}\NormalTok{,}\DecValTok{1}\NormalTok{,}\DecValTok{0}\NormalTok{), }\DataTypeTok{ncol =} \DecValTok{3}\NormalTok{)}

\NormalTok{A}\OperatorTok\NormalTok{A }\OperatorTok{-}\StringTok{ }\NormalTok{A }\OperatorTok{-}\StringTok{ }\DecValTok{2}\OperatorTok{*}\NormalTok{I}
\end{Highlighting}
\end{Shaded}

\begin{verbatim}
##      [,1] [,2] [,3]
## [1,]    0    0    0
## [2,]    0    0    0
## [3,]    0    0    0
\end{verbatim}

\hypertarget{calcular-la-matriz-inversa-de}{%
\paragraph{5. Calcular la matriz inversa
de:}\label{calcular-la-matriz-inversa-de}}

\[ A = 
 \begin{pmatrix}
1 & -1& 0 \\
0 & 1 & 0 \\
2 & 0 & 1\\
\end{pmatrix}
\] puede utilizar la función solve() de R

\begin{Shaded}
\begin{Highlighting}[]
\NormalTok{A <-}\StringTok{ }\KeywordTok{matrix}\NormalTok{(}\KeywordTok{c}\NormalTok{(}\DecValTok{1}\NormalTok{,}\DecValTok{0}\NormalTok{,}\DecValTok{2}\NormalTok{,}\OperatorTok{-}\DecValTok{1}\NormalTok{,}\DecValTok{1}\NormalTok{,}\DecValTok{0}\NormalTok{,}\DecValTok{0}\NormalTok{,}\DecValTok{0}\NormalTok{,}\DecValTok{1}\NormalTok{), }\DataTypeTok{ncol =} \DecValTok{3}\NormalTok{)}
\KeywordTok{solve}\NormalTok{(A)}
\end{Highlighting}
\end{Shaded}

\begin{verbatim}
##      [,1] [,2] [,3]
## [1,]    1    1    0
## [2,]    0    1    0
## [3,]   -2   -2    1
\end{verbatim}

\hypertarget{resolver-en-forma-matricial-el-sistema}{%
\paragraph{6. Resolver, en forma matricial, el
sistema:}\label{resolver-en-forma-matricial-el-sistema}}

\[ 
\begin{cases}
    x + y +z = 6 \\ 
   x + 2y+ 5z= 12 \\
   x+ 4y+ 25z= 362 \\
  \end{cases}
\]

\begin{Shaded}
\begin{Highlighting}[]
\NormalTok{A <-}\StringTok{ }\KeywordTok{matrix}\NormalTok{(}\KeywordTok{c}\NormalTok{(}
        \DecValTok{1}\NormalTok{,}\DecValTok{1}\NormalTok{,}\DecValTok{1}\NormalTok{,}
        \DecValTok{1}\NormalTok{,}\DecValTok{2}\NormalTok{,}\DecValTok{5}\NormalTok{,}
        \DecValTok{1}\NormalTok{,}\DecValTok{4}\NormalTok{,}\DecValTok{25}
\NormalTok{         ), }\DataTypeTok{ncol =} \DecValTok{3}\NormalTok{, }\DataTypeTok{byrow =}\NormalTok{ T)}
\NormalTok{B <-}\StringTok{ }\KeywordTok{c}\NormalTok{(}\DecValTok{6}\NormalTok{,}\DecValTok{12}\NormalTok{,}\DecValTok{25}\NormalTok{)}

\KeywordTok{solve}\NormalTok{(A,B)}
\end{Highlighting}
\end{Shaded}

\begin{verbatim}
## [1] 0.25000000 5.66666667 0.08333333
\end{verbatim}

\hypertarget{diagonalizacion-y-valores-singulares}{%
\section{Diagonalización y valores
singulares}\label{diagonalizacion-y-valores-singulares}}

\hypertarget{sean}{%
\paragraph{1.(∗)Sean\ldots{}}\label{sean}}

\begin{enumerate}
\def\labelenumi{\alph{enumi})}
\tightlist
\item
  Probar queues un vector propio deA. ¿Cual es su valor propio
  correspondiente?
\item
  Comprobar queλu, conλun escalar no nulo, también es vector propio deA.
\item
  Probar queves un vector propio deA. ¿Cual es su valor propio
  correspondiente?
\item
  ¿u+ves un vector propio deA?
\end{enumerate}

\hypertarget{para-las-siguientes-matrices-determinar}{%
\paragraph{2. Para las siguientes matrices
determinar:}\label{para-las-siguientes-matrices-determinar}}

\begin{enumerate}
\def\labelenumi{\alph{enumi})}
\tightlist
\item
  el polinomio característico,
\item
  los valores propios,
\item
  vectores propios para cada valor propio,
\item
  (∗) la multiplicidad de cada valor propio y el número de vectores
  propios independientes asociados a cada valor propio.
\end{enumerate}

Se puede utilizar la función eigen() de R.

\begin{Shaded}
\begin{Highlighting}[]
\NormalTok{A <-}\StringTok{ }\KeywordTok{matrix}\NormalTok{(}\KeywordTok{c}\NormalTok{(}\DecValTok{0}\NormalTok{,}\DecValTok{0}\NormalTok{,}\DecValTok{1}\NormalTok{,}\DecValTok{0}\NormalTok{,}\DecValTok{1}\NormalTok{,}\DecValTok{0}\NormalTok{,}\DecValTok{1}\NormalTok{,}\DecValTok{0}\NormalTok{,}\DecValTok{0}\NormalTok{), }\DataTypeTok{ncol =} \DecValTok{3}\NormalTok{)}
\NormalTok{B <-}\StringTok{ }\KeywordTok{matrix}\NormalTok{(}\KeywordTok{c}\NormalTok{(}\DecValTok{1}\NormalTok{,}\DecValTok{0}\NormalTok{,}\DecValTok{0}\NormalTok{,}\DecValTok{1}\NormalTok{,}\DecValTok{2}\NormalTok{,}\DecValTok{3}\NormalTok{,}\DecValTok{0}\NormalTok{,}\DecValTok{0}\NormalTok{,}\DecValTok{3}\NormalTok{), }\DataTypeTok{ncol =} \DecValTok{3}\NormalTok{)}
\NormalTok{D <-}\StringTok{ }\KeywordTok{matrix}\NormalTok{(}\KeywordTok{c}\NormalTok{(}\DecValTok{0}\NormalTok{,}\DecValTok{0}\NormalTok{,}\DecValTok{4}\NormalTok{,}\DecValTok{0}\NormalTok{,}\DecValTok{2}\NormalTok{,}\DecValTok{0}\NormalTok{,}\DecValTok{1}\NormalTok{,}\DecValTok{0}\NormalTok{,}\DecValTok{0}\NormalTok{), }\DataTypeTok{ncol =} \DecValTok{3}\NormalTok{)}
\NormalTok{E <-}\StringTok{ }\KeywordTok{matrix}\NormalTok{(}\KeywordTok{c}\NormalTok{(}\DecValTok{0}\NormalTok{,}\DecValTok{2}\NormalTok{,}\DecValTok{0}\NormalTok{,}\DecValTok{2}\NormalTok{,}\DecValTok{0}\NormalTok{,}\DecValTok{0}\NormalTok{,}\DecValTok{0}\NormalTok{,}\DecValTok{0}\NormalTok{,}\DecValTok{3}\NormalTok{), }\DataTypeTok{ncol =} \DecValTok{3}\NormalTok{)}
\end{Highlighting}
\end{Shaded}

\begin{enumerate}
\def\labelenumi{\alph{enumi})}
\tightlist
\item
  \[
  \begin{pmatrix}
  0-\lambda & 0 & 1 \\
  0 & 1-\lambda & 0 \\
  1 & 0 & 0-\lambda
  \end{pmatrix} = - \lambda³ + \lambda² + \lambda
  \] \[
  \begin{pmatrix}
  1-\lambda & 1 & 0 \\
  0 & 2-\lambda & 0 \\
  0 & 0 & 3-\lambda 
  \end{pmatrix} =  \lambda²- 3\lambda + 2
  \]
\end{enumerate}

\[
\begin{pmatrix}
0-\lambda & 0 & 1 \\
0 & 2-\lambda & 0 \\
4 & 0 & 0-\lambda 
\end{pmatrix} = - \lambda³ + 2\lambda² + 4\lambda -8
\]

\[
\begin{pmatrix}
0-\lambda & 2 & 0 \\
2 & 0-\lambda & 0 \\
0 & 0 & 3-\lambda 
\end{pmatrix} = -\lambda³ + 3\lambda² + 4\lambda - 12
\]

\begin{Shaded}
\begin{Highlighting}[]
\CommentTok{# b) }
\KeywordTok{eigen}\NormalTok{(A)}\OperatorTok{$}\NormalTok{values}
\end{Highlighting}
\end{Shaded}

\begin{verbatim}
## [1]  1  1 -1
\end{verbatim}

\begin{Shaded}
\begin{Highlighting}[]
\KeywordTok{eigen}\NormalTok{(B)}\OperatorTok{$}\NormalTok{values}
\end{Highlighting}
\end{Shaded}

\begin{verbatim}
## [1] 3 2 1
\end{verbatim}

\begin{Shaded}
\begin{Highlighting}[]
\KeywordTok{eigen}\NormalTok{(D)}\OperatorTok{$}\NormalTok{values}
\end{Highlighting}
\end{Shaded}

\begin{verbatim}
## [1]  2  2 -2
\end{verbatim}

\begin{Shaded}
\begin{Highlighting}[]
\KeywordTok{eigen}\NormalTok{(E)}\OperatorTok{$}\NormalTok{values}
\end{Highlighting}
\end{Shaded}

\begin{verbatim}
## [1]  3  2 -2
\end{verbatim}

\begin{Shaded}
\begin{Highlighting}[]
\CommentTok{# c)}
\KeywordTok{eigen}\NormalTok{(A)}\OperatorTok{$}\NormalTok{vector}
\end{Highlighting}
\end{Shaded}

\begin{verbatim}
##      [,1]       [,2]       [,3]
## [1,]    0 -0.7071068  0.7071068
## [2,]   -1  0.0000000  0.0000000
## [3,]    0 -0.7071068 -0.7071068
\end{verbatim}

\begin{Shaded}
\begin{Highlighting}[]
\KeywordTok{eigen}\NormalTok{(B)}\OperatorTok{$}\NormalTok{vector}
\end{Highlighting}
\end{Shaded}

\begin{verbatim}
##      [,1]       [,2] [,3]
## [1,]    0  0.3015113    1
## [2,]    0  0.3015113    0
## [3,]    1 -0.9045340    0
\end{verbatim}

\begin{Shaded}
\begin{Highlighting}[]
\KeywordTok{eigen}\NormalTok{(D)}\OperatorTok{$}\NormalTok{vector}
\end{Highlighting}
\end{Shaded}

\begin{verbatim}
##           [,1] [,2]       [,3]
## [1,] 0.4472136    0 -0.4472136
## [2,] 0.0000000    1  0.0000000
## [3,] 0.8944272    0  0.8944272
\end{verbatim}

\begin{Shaded}
\begin{Highlighting}[]
\KeywordTok{eigen}\NormalTok{(E)}\OperatorTok{$}\NormalTok{vector}
\end{Highlighting}
\end{Shaded}

\begin{verbatim}
##      [,1]      [,2]       [,3]
## [1,]    0 0.7071068  0.7071068
## [2,]    0 0.7071068 -0.7071068
## [3,]    1 0.0000000  0.0000000
\end{verbatim}

\hypertarget{si-s-010101101-es-el-conjunto-de-vectores-propios-para-los-valores-propios-111-hallar-la-matriz-a-correspondiente.}{%
\paragraph{\texorpdfstring{3. Si \$ S = \{(0,1,0),(1,0,1),(−1,0,1)\} \$
es el conjunto de vectores propios, para los valores propios \(1,1,−1\),
hallar la matriz A
correspondiente.}{3. Si \$ S = \{(0,1,0),(1,0,1),(−1,0,1)\} \$ es el conjunto de vectores propios, para los valores propios 1,1,−1, hallar la matriz A correspondiente.}}\label{si-s-010101101-es-el-conjunto-de-vectores-propios-para-los-valores-propios-111-hallar-la-matriz-a-correspondiente.}}

\begin{Shaded}
\begin{Highlighting}[]
\NormalTok{V <-}\StringTok{ }\KeywordTok{cbind}\NormalTok{(}\KeywordTok{c}\NormalTok{(}\DecValTok{0}\NormalTok{,}\DecValTok{1}\NormalTok{,}\DecValTok{0}\NormalTok{), }\KeywordTok{c}\NormalTok{(}\DecValTok{1}\NormalTok{,}\DecValTok{0}\NormalTok{,}\DecValTok{1}\NormalTok{), }\KeywordTok{c}\NormalTok{(}\OperatorTok{-}\DecValTok{1}\NormalTok{,}\DecValTok{0}\NormalTok{,}\DecValTok{1}\NormalTok{))}
\NormalTok{L <-}\StringTok{ }\KeywordTok{diag}\NormalTok{(}\KeywordTok{c}\NormalTok{(}\DecValTok{1}\NormalTok{,}\DecValTok{1}\NormalTok{,}\OperatorTok{-}\DecValTok{1}\NormalTok{))}
\NormalTok{A <-}\StringTok{ }\NormalTok{V }\OperatorTok\StringTok{ }\NormalTok{L }\OperatorTok\StringTok{ }\KeywordTok{solve}\NormalTok{(V)}
\NormalTok{A}
\end{Highlighting}
\end{Shaded}

\begin{verbatim}
##      [,1] [,2] [,3]
## [1,]    0    0    1
## [2,]    0    1    0
## [3,]    1    0    0
\end{verbatim}

\begin{Shaded}
\begin{Highlighting}[]
\CommentTok{# checkeamos pero no coincide?!?!}
\KeywordTok{eigen}\NormalTok{(A)}
\end{Highlighting}
\end{Shaded}

\begin{verbatim}
## eigen() decomposition
## $values
## [1]  1  1 -1
## 
## $vectors
##      [,1]       [,2]       [,3]
## [1,]    0 -0.7071068  0.7071068
## [2,]   -1  0.0000000  0.0000000
## [3,]    0 -0.7071068 -0.7071068
\end{verbatim}

\hypertarget{comprobar-las-siguientes-propiedades-con-algun-ejemplo}{%
\paragraph{4.(∗)Comprobar las siguientes propiedades con algún
ejemplo:}\label{comprobar-las-siguientes-propiedades-con-algun-ejemplo}}

\hypertarget{mediante-la-diagonalizacion-de-la-matriz}{%
\paragraph{5. Mediante la diagonalización de la
matriz}\label{mediante-la-diagonalizacion-de-la-matriz}}

\$\$ A =

\begin{pmatrix}
1 & 2  \\
3 & 2  \\

\end{pmatrix}

\$\$ Calcular \(A^7\)

\hypertarget{dada-la-matriz-de-covarianzas}{%
\paragraph{6. Dada la matriz de
covarianzas}\label{dada-la-matriz-de-covarianzas}}

Hallar una matrizΣ−1/2tal queΣ−1/2Σ−1/2=Σ−1.

\hypertarget{hallar-la-descomposicion-en-valores-singulares-de-la-matriz}{%
\paragraph{7. Hallar la descomposición en valores singulares de la
matriz}\label{hallar-la-descomposicion-en-valores-singulares-de-la-matriz}}

\[
A=
\begin{pmatrix}
2&4\\
1&3\\
0&0\\
0&0\\
\end{pmatrix}
\]

\begin{Shaded}
\begin{Highlighting}[]
\NormalTok{A <-}\StringTok{ }\KeywordTok{matrix}\NormalTok{(}\KeywordTok{c}\NormalTok{(}\DecValTok{2}\NormalTok{,}\DecValTok{1}\NormalTok{,}\DecValTok{0}\NormalTok{,}\DecValTok{0}\NormalTok{,}\DecValTok{4}\NormalTok{,}\DecValTok{3}\NormalTok{,}\DecValTok{0}\NormalTok{,}\DecValTok{0}\NormalTok{), }\DataTypeTok{ncol =} \DecValTok{2}\NormalTok{)}

\KeywordTok{svd}\NormalTok{(A)}
\end{Highlighting}
\end{Shaded}

\begin{verbatim}
## $d
## [1] 5.4649857 0.3659662
## 
## $u
##            [,1]       [,2]
## [1,] -0.8174156 -0.5760484
## [2,] -0.5760484  0.8174156
## [3,]  0.0000000  0.0000000
## [4,]  0.0000000  0.0000000
## 
## $v
##            [,1]       [,2]
## [1,] -0.4045536 -0.9145143
## [2,] -0.9145143  0.4045536
\end{verbatim}

\hypertarget{calcular-el-rango-de-la-matriz}{%
\paragraph{8.Calcular el rango de la
matriz}\label{calcular-el-rango-de-la-matriz}}

\begin{Shaded}
\begin{Highlighting}[]
\NormalTok{A <-}\StringTok{ }\KeywordTok{matrix}\NormalTok{(}\KeywordTok{c}\NormalTok{(}\DecValTok{2}\NormalTok{,}\DecValTok{3}\NormalTok{,}\OperatorTok{-}\DecValTok{1}\NormalTok{,}\DecValTok{3}\NormalTok{,}\DecValTok{0}\NormalTok{,}
              \DecValTok{1}\NormalTok{,}\DecValTok{2}\NormalTok{,}\DecValTok{1}\NormalTok{,}\OperatorTok{-}\DecValTok{2}\NormalTok{,}\DecValTok{1}\NormalTok{,}
              \DecValTok{3}\NormalTok{,}\DecValTok{5}\NormalTok{,}\DecValTok{0}\NormalTok{,}\DecValTok{1}\NormalTok{,}\DecValTok{1}\NormalTok{,}
              \DecValTok{2}\NormalTok{,}\DecValTok{1}\NormalTok{,}\OperatorTok{-}\DecValTok{7}\NormalTok{,}\DecValTok{17}\NormalTok{,}\OperatorTok{-}\DecValTok{4}\NormalTok{), }\DataTypeTok{ncol =} \DecValTok{4}\NormalTok{)}

\NormalTok{matrix.rank <-}\StringTok{ }\ControlFlowTok{function}\NormalTok{(A, }\DataTypeTok{eps=}\KeywordTok{sqrt}\NormalTok{(.Machine}\OperatorTok{$}\NormalTok{double.eps))\{}
\NormalTok{  sv. <-}\StringTok{ }\KeywordTok{abs}\NormalTok{(}\KeywordTok{svd}\NormalTok{(A)}\OperatorTok{$}\NormalTok{d)}
  \KeywordTok{sum}\NormalTok{((sv.}\OperatorTok{/}\KeywordTok{max}\NormalTok{(sv.))}\OperatorTok{>}\NormalTok{eps)}
\NormalTok{  \}}


\NormalTok{sv. <-}\StringTok{ }\KeywordTok{abs}\NormalTok{(}\KeywordTok{svd}\NormalTok{(A)}\OperatorTok{$}\NormalTok{d)}

\CommentTok{# una definición de rango es que corresponde a los valores singulares diferentes a 0.}
\CommentTok{# calcula el numero de valores singulares != 0 o en este caso, }
\CommentTok{# el menor flotante posible para la maquina}



\KeywordTok{matrix.rank}\NormalTok{(A)}
\end{Highlighting}
\end{Shaded}

\begin{verbatim}
## [1] 2
\end{verbatim}

\hypertarget{probar-que-para-la-matriz-dos-inversas-generalizadas-son}{%
\paragraph{9. (∗) Probar que para la matriz dos inversas generalizadas
son}\label{probar-que-para-la-matriz-dos-inversas-generalizadas-son}}

\hypertarget{hallar-una-inversa-generalizada-de-la-matriz-mediante-la-inversa-del-menor-de-rango-maximo.}{%
\paragraph{10. (∗) Hallar una inversa generalizada de la matriz mediante
la inversa del menor de rango
máximo.}\label{hallar-una-inversa-generalizada-de-la-matriz-mediante-la-inversa-del-menor-de-rango-maximo.}}

\hypertarget{determinar-la-inversa-de-moore-penrose-de-la-matriz}{%
\paragraph{11. (∗) Determinar la inversa de Moore-Penrose de la
matriz}\label{determinar-la-inversa-de-moore-penrose-de-la-matriz}}

Utilizar la funciónginv()del paqueteMASSdeR

a


\end{document}
